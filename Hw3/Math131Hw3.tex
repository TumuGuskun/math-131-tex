\documentclass[12pt,letterpaper,boxed,cm]{hmcpset}

\usepackage[margin=1in]{geometry}
\usepackage{mathtools}
\usepackage{mathrsfs}
\usepackage{graphicx}
\usepackage{cases}


\name{~}
\class{Math 131}
\assignment{Homework 3}
\duedate{2/13/17}

\newcommand{\pn}[1]{\left( #1 \right)}
\newcommand{\abs}[1]{\left| #1 \right|}
\newcommand{\bk}[1]{\left[ #1 \right]}
\newcommand{\set}[1]{\left\{#1\right\}}
\newcommand{\R}[0]{\mathbb{R}}
\newcommand{\Q}[0]{\mathbb{Q}}
\newcommand{\Z}[0]{\mathbb{Z}}

\begin{document}
\problemlist{1, 2, 3}
For all three of the problems below, view $\R$ as a metric space with the usual metric.\\
\begin{problem}[(1)]
    Prove or disprove the following: There is only one open set in $\R$ that contains all of
    $\Q$, and it is the set $\R$.
\end{problem}

\begin{solution}
    \vfill
\end{solution}
\newpage

\begin{problem}[(2)]
    How many subsets of $\R$ are both open and closed?
\end{problem}

\begin{solution}
    \vfill
\end{solution}
\newpage

\begin{problem}[(3)]
    Rudin uses Theorem 1.20 to say that ``$\Q$ is dense in $\R$.'' But is $\Q$ still dense in $\R$
    if we instead use the definition for ``dense'' given in Definition 2.18? What is the
    relationship between these two definitions of ``dense'' when applied to $\Q$ in $\R$? Does
    each one imply the other?
\end{problem}

\begin{solution}
    \vfill
\end{solution}
\newpage
\end{document}