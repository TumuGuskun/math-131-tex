\documentclass[12pt,letterpaper,boxed,cm]{hmcpset}

\usepackage[margin=1in]{geometry}
\usepackage{mathtools}
\usepackage{mathrsfs}
\usepackage{graphicx}
\usepackage{cases}
\usepackage{enumitem}
\usepackage{wasysym}

\name{~}
\class{Math 131}
\assignment{Homework 5}
\duedate{3/6/17}

\newcommand{\pn}[1]{\left( #1 \right)}
\newcommand{\abs}[1]{\left| #1 \right|}
\newcommand{\bk}[1]{\left[ #1 \right]}
\newcommand{\set}[1]{\left\{#1\right\}}
\newcommand{\R}[0]{\mathbb{R}}

\begin{document}
\problemlist{1, 2, 3}

\begin{problem}[(1)]
    Prove that the sequence $\set{p_n}$ converges to $p$ if and only if every subsequence of $\set{p_n}$ converges to $p$. (See Rudin, p. 51.)
\end{problem}

\begin{solution}
    \vfill
\end{solution}
\newpage

\begin{problem}[(2)]
    Let $\set{x_n}$ and $\set{s_n}$ be sequences in $\R$. Suppose $0 \le x_n \le s_n$ for $n \ge N$, where $N$ is some fixed number. Prove that if $s_n \rightarrow 0$, then $x_n \rightarrow 0$. (See Rudin, p. 57.)
\end{problem}

\begin{solution}
    \vfill
\end{solution}
\newpage

\begin{problem}[(3)]
    Let $x$ be a real number such that $\abs{x} < 1$. Prove that if $\abs{x^n} \rightarrow 0$, then $x^n \rightarrow 0$. (We used this fact in our proof of part (e) of Theorem 3.20.)
\end{problem}

\begin{solution}
    \vfill
\end{solution}
\newpage
\end{document}
