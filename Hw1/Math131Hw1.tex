\documentclass[12pt,letterpaper,boxed,cm]{hmcpset}

\usepackage[margin=1in]{geometry}
\usepackage{mathtools}
\usepackage{mathrsfs}
\usepackage{graphicx}
\usepackage{cases}
\usepackage{dsfont}
\usepackage{braket}
\usepackage[commonsets=true]{skmath}

\name{~}
\class{Math 131}
\assignment{Homework 1}
\duedate{1/30/17}

\newcommand{\pn}[1]{\left( #1 \right)}
\renewcommand{\abs}[1]{\left| #1 \right|}
\newcommand{\bk}[1]{\left[ #1 \right]}

\begin{document}
\problemlist{1, 2, 3}

\begin{problem}[(1)]
    Prove that our definitions for the sum and product of two rational numbers are well defined. In other words, show that they do not depend on the choice of representatives for the rational numbers being added or multiplied.
\end{problem}

\begin{solution}
    \vfill
\end{solution}
\newpage

\begin{problem}[(2)]
    Let $\displaystyle \frac{a}{b},\frac{c}{d}\in \Q$. Prove of disprove the following statement: $\displaystyle \frac{a}{b}=\frac{c}{d}$ if and only if there exists a nonzero integer $m\in\Z$ such that $ma=c$ and $mb=d$.
\end{problem}

\begin{solution}
    \vfill
\end{solution}
\newpage

\begin{problem}[(3)]
    Consider the following subset of the real numbers:
    \[
        \Q\pn{\sqrt{2}}=\set{r+s \sqrt{2}: r,s\in\Q}.
    \]
    Prove that $\Q\pn{\sqrt{2}}$ forms a field under the usual addition and multiplication of real numbers. (You may use the fact that the real numbers form a field.)
\end{problem}

\begin{solution}
    \vfill
\end{solution}
\end{document}